% Parametri che modificano il file main.tex
% Le uniche parti da cambiare su main.tex sono:
% - vari \vspace tra sezioni
% - tabella azioni da intraprendere
% - sezione altro

\def\data{2023-11-07}
\def\oraInizio{17.00}
\def\oraFine{19.00}
\def\luogo{Piattaforma Discord}

\def\tipoVerb{Interno} % Interno - Esterno

\def\nomeResp{Feltrin E.} % Cognome N.
\def\nomeVer{Campese M.} % Cognome N.
\def\nomeSegr{Orlandi G.} % Cognome N.

\def\nomeAzienda{Azzurro Digitale}
\def\firmaAzienda{azzurrodigitale.png}
\def\firmaResp{emanuele.png} % nome Responsabile

\def\listaPartInt{
Bresolin G.,
Campese M.,
Ciriolo I.,
Dugo A.,
Feltrin E.,
Michelon R.,
Orlandi G.
}

\def\listaPartEst{
Azzurro B.,
Digitale C.,
}

\def\listaRevisioneAzioni{
{Esame del \textit{way of working} adottato finora alla luce delle osservazioni fatte dal professore alla candidatura;},
{Revisione dei documenti consegnati alla Candidatura che necessitano di modifiche correttive;},
{Controllo delle Azioni nei verbali con seguente aggiunta della colonna con identificativo per relazionarle alle \textit{issue} di \textit{GitHub}. }
}

\def\listaOrdineGiorno{
{Assegnazione ruoli riunione;},
{Riesame dei ruoli e della loro assegnazione;},
{Rettifiche al \textit{Way of Working} del gruppo;},
%{Ruoli da definire per il proseguo del progetto nei prossimi 11 giorni;},
{Considerazioni sulle automazioni da introdurre;},
%{Decisioni riguardanti organizzazione del \textit{repository} \textit{GitHub};},
{Vaglio dei dubbi da chiarire in futuro.}
}

\def\listaDiscussioneInterna {
{Assegnazione ruoli riunione:
    \begin{itemize}
        \item Responsabile: Feltrin E.;
        \item Segretario di riunione: Orlandi G;
        \item Responsabile della revisione: Campese M.
    \end{itemize}},
{Dibattito su quali risorse risultano fondamentali nelle prime fasi del progetto e quanto i ruoli di Amministratore e di Analista siano cruciali nell'impostazione del lavoro nelle prime settimane del progetto assegnato;},
{Valutazione di eventuali modifiche alle Norme di progetto per migliorare il nostro \textit{Way of Working}: \begin{itemize}
        \item Aggiungere un collegamento all'id delle \textit{issue} nella tabella delle azioni presente nei verbali;
        \item Modificare l'ordine del registro delle versioni a partire dalla versione più recente.
        \item Parametrizzare i template dei documenti;
    \end{itemize}},
{In seguito alle valutazioni delle Candidature del 2023-11-06, si percepisce il bisogno di:
    \begin{itemize}
    \item Introdurre automazione nel progetto, in particolare, viene evidenziato che il versionamento non deve essere eseguito manualmente;
    \item Aggiornare l'approccio al versionamento della documentazione;
    \item Aggiungere la versione al nome di ciascun documento versionato;
    \item Tutti i documenti che non sono verbali devono presentare un registro delle versioni.
    \end{itemize}},
{Dubbi da risolvere:
\begin{itemize}
    \item Argomenti da trattare con l'azienda;
    \item Come organizzarsi se l'azienda fornisce un altro \textit{repository} da integrare;
    \item Decidere se iniziare una prima bozza dell'analisi dei requisiti o se cominciare a studiare le tecnologie proposte dall'azienda;
    \item Quantificare il numero di \textit{milestone} necessarie per l'organizzazione delle \textit{issue}.
\end{itemize}
}
}

\newcommand{\domris}[2]{\textbf{#1}\\#2}

\def\listaDiscussioneEsterna {
\domris
{Domanda 1
}
{Riposta 1;
},
\domris
{Domanda 2
}
{Risposta 2;
},
\domris
{Domanda 3
}
{Risposta 3.
}
}

\def\listaDecisioni {
{Il team ha deciso di adottare, almeno inizialmente, sprint dalla lunghezza di due settimane;},
{Il team ha deciso di aprire tre milestone: "RTB-Amministrazione workspace","RTB-Documentazione" e "RTB-Analisi dei requisiti";},
{La \textit{pull request} viene sollevata da colui che produce il contenuto da integrare nel repository e viene risolta dal verificatore;},
{Parametrizzazione dei template dei documenti;},
{Le versioni presenteranno il formato vX.Y.Z(build) e non potrà esistere la versione v0.0.0(0);},
{Introdurre la versione di progetto per cui i documenti risultano essere validi all'interno del \href{https://sweetcode-team.github.io/}{\textcolor{black}{{\textit{GitHub.io}}}} del team;},
{Introduzione di automazioni per la gestione del versionamento;},
{Creazione template issue;},
{Ogni componente deve familiarizzare con le tecnologie del progetto;},
{Eliminare il \textit{repository} privato da \textit{GitHub}.}
}