% Parametri che modificano il file main.tex
% Le uniche parti da cambiare su main.tex sono:
% - vari \vspace tra sezioni
% - tabella azioni da intraprendere
% - sezione altro

\def\data{2023-02-03}
\def\oraInizio{17:00}
\def\oraFine{18:00}
\def\luogo{Piattaforma Discord}

\def\tipoVerb{Interno} % Interno - Esterno

\def\nomeResp{Bresolin G.} % Cognome N.
\def\nomeVer{Feltrin E.} % Cognome N.
\def\nomeSegr{Michelon R.} % Cognome N.

\def\nomeAzienda{Azzurro Digitale}
\def\firmaAzienda{azzurrodigitale.png}
\def\firmaResp{firmaBresolin.png} % nome Responsabile

\def\listaPartInt{
Bresolin G.,
Campese M.,
Dugo A.,
Feltrin E.,
Michelon R.
}

\def\listaPartEst{
Azzurro B.,
Digitale C.,
}

% Se nessuna revisione: \def\listaRevisioneAzioni {x}
\def\listaRevisioneAzioni {x}

\def\listaOrdineGiorno {
{Conferma dei ruoli assegnati per il settimo sprint;},
{Discussione impegni universitari;},
{Discussione ore responsabile;},
{Discussione correzzioni suggerite dal prof. Cardin R. al documento Analisi dei requisiti;},
{Discussione migliorie da introdurre al documento Piano di qualifica;},
{Pianficazione invio mail di candidatura al colloquio per la seconda fase di revisione RTB;},
{Discussione progettazione sketch delle interfacce.}
}

\def\listaDiscussioneInterna {
{I ruoli che i membri assumono all'inizio del settimo sprint sono i seguenti:
\begin{itemize}
\item Amministratore: Michelon R.;
\item Analista: Dugo A.;
\item Progettista: Ciriolo I., Campese M.;
\item Programmatori: Orlandi G.;
\item Responsabile: Bresolin G.;
\item Verificatore: Feltrin E.
\end{itemize}},
{Il team ha revisionato ulteriormente le correzzioni proposte dal prof. Cardin R. e, a seguito di una riflessione sul come applicare tali suggirimenti, ha assegnato il task a Dugo A.;},
{Il gruppo ha discusso quali fossero gli impegni universitari rimanenti a causa della sessione invernale e, seppur una parte del team sia ancora occupato per alcuni giorni, la maggioranza si è resa disponibile
a tornare operativa a regime fin da subito, rendendo la pianificazione elaborata in passato per il presente sprint fattibile;},
{Il team prende in considerazione la possibilità di ridurre le ore destinate al ruolo di responsabile future in quanto la ridotta pianificazione di ore preventivate per tale ruolo, seppur a parità di task richiesti,
non ha comportato un problema: partendo da questa riflessione, il gruppo deciderà in fase di stesura della retrospettiva dello sprint se destinare parte delle risorse economiche destinate alle ore da
responsabile per una pianificazione con più ore per il ruolo di progettista e programmatore, i quali ricopriranno un aspetto fondamentale negli sprint a seguire;},
{A seguito di una riflessione condivisa tra il gruppo, è emersa la necessità di apportare migliorie al documento (Piano di Qualifica v1.10.4(12)), approfondendo le considerazioni sulle metriche;},
{Il team si pone l'obiettivo di inviare la mail al prof. Vardanega T. per la candidatura alla seconda fase della revisione RTB entro la fine del presente sprint;},
{Come richiesto dall'azienda proponente, il gruppo si impegnerà nella produzione degli sketch per le interfacce attraverso \textit{Figma}.}
}



% Se nessuna decisione: \def\listaDecisioni {x}
\def\listaDecisioni{
{Il team ha deciso di presentare la candidatura al colloquio per la seconda fase di revisione RTB;},
{Il team ha deciso di applicare le correzioni suggerite dal prof. Cardin R. all'Analisi dei requisiti;},
{Il team ha deciso di produrre gli sketch richiesti dall'azienda proponente attraverso \textit{Figma}.}
}