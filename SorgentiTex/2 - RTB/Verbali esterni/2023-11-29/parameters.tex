% Parametri che modificano il file main.tex
% Le uniche parti da cambiare su main.tex sono:
% - vari \vspace tra sezioni
% - tabella azioni da intraprendere
% - sezione altro

\def\data{2023-11-29}
\def\oraInizio{14:00}
\def\oraFine{15:00}
\def\luogo{Google Meet}

\def\tipoVerb{Esterno} % Interno - Esterno

\def\nomeResp{Ciriolo I.} % Cognome N.
\def\nomeVer{Orlandi G.} % Cognome N.
\def\nomeSegr{Ciriolo I.} % Cognome N.

\def\nomeAzienda{Azzurro Digitale}
\def\firmaAzienda{azzurrodigitale.png}
\def\firmaResp{irene.png} % nome Responsabile

\def\listaPartInt{
Bresolin G.,
Campese M.,
Ciriolo I.,
Dugo A.,
Feltrin E.,
Michelon R.,
Orlandi G.
}

\def\listaPartEst{
Bendotti E.,
Davanzo C.
}

% Se nessuna revisione: \def\listaRevisioneAzioni {x}
\def\listaRevisioneAzioni {x}

\def\listaOrdineGiorno {
{Presentazione dei PoC creati dal team con feedback da parte dell'azienda.}
}

\def\listaDiscussioneInterna {
{Discussione 1;},
{Discussione 2;},
}

\newcommand{\domris}[2]{\textbf{#1}\\#2}

\def\listaDiscussioneEsterna {
\domris
{PoC proposto da Feltrin E.
}{
Il sistema presentato dal programmatore Feltrin E. utilizza tecnologie di \textit{OpenAI} per creare gli \textit{embeddings} a partire da documenti forniti dall’utente e il modello \textit{gpt 3.5-turbo} di \textit{OpenAI} per la generazione della risposta alla domanda posta dall’utente.
Usa inoltre il vector store \textit{Alloy} per il salvataggio dei vettori in locale.\\
Le tecnologie proposte, in particolare quelle di \textit{OpenAI}, sono soggette a tariffe.
Durante la codifica di questo PoC, è stato utilizzato il credito iniziale gratuito dell’account di un componente del gruppo. Dopo la presentazione, l’azienda conferma al team che le risposte del sistema sono sufficientemente elaborate e specifiche. Sono infatti in grado di soddisfare le richieste dell’utente, mantenendo inoltre traccia della storia della conversazione e del contesto.
},
\domris
{PoC proposto da Michelon R.
}{
Il secondo PoC presentato, a differenza del primo, utilizza esclusivamente tecnologie open source.
Vengono infatti utilizzati modelli di \textit{embeddings} e \textit{llm} di \textit{HuggingFace}, una piattaforma che raccoglie e fornisce strumenti per lo sviluppo di applicazioni AI-based.
In particolare vengono presi in considerazione i modelli \textit{sentence-transformers/all-MiniLM-L6-v2} e \textit{google/flan-t5-xxl}.
Viene inoltre scelto \textit{ChromaDB} come vector store locale.\\
La scelta delle tecnologie era vincolata dalla richiesta dell’azienda di rendere disponibili opzioni locali per garantire la privacy dei dati dell’utente.
Visto il ridotto tempo a disposizione, si è optato per un’interfaccia grafica predefinita, fornita dal framework \textit{StreamLit}. Questa scelta ha permesso al team di concentrarsi maggiormente sulle funzionalità più critiche, posticipando l’esplorazione del frontend a PoC successivi.
Date le limitate capacità computazionali dei dispositivi in cui viene ospitato il sistema, la scelta dei modelli ha influenzato negativamente le performance del sistema. I modelli locali forniscono risposte non sempre pertinenti e ben elaborate.
}
}

% Se nessuna decisione: \def\listaDecisioni {x}
\def\listaDecisioni {
{Per la successiva riunione l'azienda richiede:
\begin{itemize}
\item Refactoring del codice proposto dal team (che per il momento non presenta struttura);
\item Presenza di un pattern architetturale;
\item Sostituzione di \textit{Streamlit} con \textit{Angular} e \textit{React} (devono essere testate entrambe le tecnologie);
\item Aggiornamento documentazione;
\item Abbellimento interfacce user.
\end{itemize}
}
}