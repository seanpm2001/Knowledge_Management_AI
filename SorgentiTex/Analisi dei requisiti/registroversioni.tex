%Registro in ordine dalla più recente alla meno recente!
{\renewcommand{\arraystretch}{1.5}
\section*{Registro delle versioni}

\begin{xltabular}{\textwidth}{c|c|c|c|X}
\label{tab:long}

\textbf{Versione} & \textbf{Data} & \quantities{\textbf{Responsabile di}\\\textbf{stesura}}& \textbf{Revisore} & \quantities{\textbf{Dettaglio e}\\\textbf{motivazioni}} \\
\endfirsthead

\textbf{Versione} & \textbf{Data} & \quantities{\textbf{Responsabile di}\\\textbf{stesura}}& \textbf{Revisore} & \quantities{\textbf{Dettaglio e}\\\textbf{motivazioni}} \\
\endhead

\multicolumn{5}{r}{{Continua nella pagina successiva}} \\
\endfoot

\endlastfoot

\hline
v1.10.0(8) & $2023-12-29$ & \quantities{Michelon R.} & Campese M. & Aggiunto riepilogo requisiti.\\
\hline
v1.10.0(7) & $2023-12-29$ & \quantities{Michelon R.} & Dugo A. & Aggiunti diagrammi use case e di attività.\\
\hline
v1.10.0(6) & $2023-12-29$ & \quantities{Michelon R.} & Dugo A. & Aggiornamento use case e requisiti.\\
\hline
v1.0.0(14) & $2023-12-06$ & \quantities{Bresolin G.} & Orlandi G. & Aggiornamento UC e requisiti presenti. Introduzione diagrammi dei casi d'uso. Aggiunta di nuovi UC e nuovi requisiti.\\
\hline
v1.0.0(5) & $2023-11-23$ & \quantities{Ciriolo I.} & Campese M. & Introduzione, Descrizione, Studio dei primi Casi d'uso, Studio dei primi requisiti.\\

    
\end{xltabular}}